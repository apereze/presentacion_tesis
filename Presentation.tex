%%%%%%%%%%%%%%%%%%%%%%%%%%%%%%%%%%%%%%%%%%%%%%%%%%%%%%%%%%%%%%%%%%%%%%%%%%%%%%%%%%%%
% IISER Thiruvananthapuram Presentation/Beamer Format
% LaTeX Template
%
% Author:
% Nikhil Alex Verghese, BS-MS 17, IISER Thiruvananthapuram
% PLEASE FORWARD ANY AND ALL SUGGESTIONS AND COMPLAINTS TO: nikhil.alexv17@alumni.iisertvm.ac.in
%
% READ ALL INSTRUCTIONS (presented as comments) IN EACH TEX FILE CAREFULLY.
%
% License:
% CC BY-NC-SA 4.0 (http://creativecommons.org/licenses/by-nc-sa/4.0/)
%
%%%%%%%%%%%%%%%%%%%%%%%%%%%%%%%%%%%%%%%%%%%%%%%%%%%%%%%%%%%%%%%%%%%%%%%%%%%%%%%%%%%%

% Refer to these super useful links for beamer tips and tricks
% http://tug.ctan.org/macros/latex/contrib/beamer/doc/beameruserguide.pdf
% https://www.overleaf.com/learn/latex/Beamer

%-----------------------------------------------------------------------------------
%	PACKAGES AND OTHER DOCUMENT CONFIGURATIONS
%-----------------------------------------------------------------------------------

\documentclass[10pt,presentation,shownotes]{beamer}
\usetheme{Warsaw}
\usecolortheme{crane} % Yellow and Blue color scheme
\usepackage{beamerthemesplit}
\usefonttheme{default}
\setbeamertemplate{caption}[numbered]
\setbeamercolor{block body example}{bg=green!12}
\setbeamertemplate{navigation symbols}{} % Uncomment to disable the bottom right navigation buttons
\setbeamercovered{transparent}
\setbeamertemplate{theorems}[numbered]
% THE FOLLOWING TEMPLATE IS FRAGILE: It is recommended you familiarise yourself with 
% a rough idea of the presentation's framework before adding custom elements.

\usepackage[utf8]{inputenc}
\usepackage[T1]{fontenc}
\usepackage{lmodern}
\usepackage[spanish]{babel}
\usepackage{csquotes}
\usepackage{mathtools,amsfonts,amssymb,setspace}

\usepackage{pstricks} % PSTricks offers an extensive collection of quick and easy macros for generating PostScript including macros for colour, graphics, pie charts, rotation, trees and overlays.
% (https://ctan.org/pkg/pstricks-base?lang=en)

\usepackage{hyperref} % hypperref package for creating reliable hyperlinks and customizations
% (https://ctan.org/pkg/hypperref?lang=en)

\usepackage{tikz} % Tikz package for drawing graphs and diagrams [XY-pic is now outdated]
% (https://ctan.org/pkg/tikz?lang=en, https://www.overleaf.com/learn/latex/TikZ_package)

%\usepackage{tikz-cd} % Commutative Diagrams with Tikz, primarily for linear algebra.
% (https://ctan.org/pkg/tikz-cd?lang=en)

%\usepackage{pgfplots} % The Pgfplots package is dependent on tikz package and is used to portray
% detailed 2D and 3D function plots from scratch as well as plot available data with a high degree
% of customization (https://www.overleaf.com/learn/latex/Pgfplots_package) 
% (https://ctan.org/pkg/pgfplots?lang=en)

%\usepackage{caption} % The captions package allows you to better control captions for floating environments like figures, etc. (https://ctan.org/pkg/caption?lang=en)

%\usepackage{booktabs} % The booktabs package allows better control over tables including ruling, width, etc. (https://ctan.org/pkg/booktabs?lang=en) 

\usepackage{booktabs}
\usepackage{multirow}
\usepackage{graphicx}
\usepackage{xcolor} % Customize colours for hyperlinks, etc.
% (https://ctan.org/pkg/xcolor?lang=en)
\hypersetup{
	colorlinks,
	linkcolor={blue!50!black},
	citecolor={blue!50!black},
	urlcolor={blue!80!black}
}

\usepackage{appendixnumberbeamer} % This resets the page count after the \appendix function is called, unlike the traditional appendix package (https://ctan.org/pkg/appendixnumberbeamer?lang=en)

\usepackage[backend=biber,style=alphabetic]{biblatex} % We use BiBLaTeX for our bibliography.
% (https://ctan.org/pkg/biblatex?lang=en)
\renewcommand*{\nameyeardelim}{\addcomma\addspace}
\addbibresource{References/ref.bib}

% The lipsum package called in the next line is just for generating dummy text throughout this template and is not necessary.
\usepackage{lipsum}

% Note: Changes to specific presentation elements usually start with /makeatletter and ends with /makeatother

% Defining an Example Block
\makeatletter
\def\th@remark{%
    \normalfont % body font
    \setbeamercolor{block title example}{bg=orange,fg=black}
    \setbeamercolor{block body example}{bg=orange!20,fg=black}
    \def\inserttheoremblockenv{exampleblock}
  }
\makeatother

\theoremstyle{remark}
\newtheorem*{remark}{Remark}

% Setting Header Properties
\setbeamertemplate{headline}
{
  \leavevmode%
  \hbox{%
  \begin{beamercolorbox}[wd=.5\paperwidth,ht=2.5ex,dp=1ex,left,leftskip=1em]{section in head/foot}%
    \usebeamerfont{subsection in head/foot}\hspace*{2ex}\insertshorttitle
  \end{beamercolorbox}%
  \begin{beamercolorbox}[wd=.5\paperwidth,ht=2.5ex,dp=1ex,center]{date in head/foot}%
    \usebeamerfont{date in head/foot}\insertshortdate{}\hspace*{2ex}
  \end{beamercolorbox}}%
  \vskip0pt%
}

% Setting Footer Properties
\makeatletter
\setbeamertemplate{footline}
{
  \leavevmode%
  \hbox{%
  \begin{beamercolorbox}[wd=.33\paperwidth,ht=2.25ex,dp=1ex,center]{author in head/foot}%
    \usebeamerfont{author in head/foot}\insertshortauthor~~\beamer@ifempty{\insertshortinstitute}{}{(\insertshortinstitute)}
  \end{beamercolorbox}%
  \begin{beamercolorbox}[wd=.34\paperwidth,ht=2.25ex,dp=1ex,center]{subsection in head/foot}%
    \usebeamerfont{section in head/foot}\hspace*{1ex}\insertsectionhead\hspace*{1ex}
  \end{beamercolorbox}%
  \begin{beamercolorbox}[wd=.33\paperwidth,ht=2.25ex,dp=1ex,right, rightskip=1em]{section in head/foot}%
    \usebeamerfont{section in head/foot}\insertsubsectionhead\hspace*{2ex}
  \end{beamercolorbox}}%
  \vskip0pt%
}
\makeatother

% Defining when a title is required for a frame
\makeatletter
\setbeamertemplate{frametitle}{
    \ifbeamercolorempty[bg]{frametitle}{}{\nointerlineskip}%
    \@tempdima=\textwidth%
    \advance\@tempdima by\beamer@leftmargin%
    \advance\@tempdima by\beamer@rightmargin%
    \begin{beamercolorbox}[sep=0.3cm,center,wd=\the\@tempdima]{frametitle}
        \usebeamerfont{frametitle}%
        \vbox{}\vskip-1ex%
        \if@tempswa\else\csname beamer@ftecenter\endcsname\fi%
        \strut\insertframetitle\strut\par%
        {%
            \ifx\insertframesubtitle\@empty%
            \else%
            {\usebeamerfont{framesubtitle}\usebeamercolor[fg]{framesubtitle}\insertframesubtitle\strut\par}%
            \fi
        }%
        \vskip-1ex%
        \if@tempswa\else\vskip-.3cm\fi% set inside beamercolorbox... evil here...
    \end{beamercolorbox}%
}
\makeatother

% This code displays a self-updating ToC at the beginning of every section.
\AtBeginSection[]
{
	\begin{frame}
	\frametitle{Flujo de la presentación}
	\tableofcontents[currentsection]
	\end{frame}
}

\makeatletter
\newcommand<>{\insertsubsectiontitle}{\frametitle{\insertsubsection}}
\let\oldbeamer@checkframetitle\beamer@checkframetitle% Store the \frametitle checking mechanism
\renewcommand<>{\subsection}{%
  \gdef\beamer@checkframetitle{% Update \frametitle checking to ...
    \insertsubsectiontitle% ...insert the section title and...
    \global\let\beamer@checkframetitle\oldbeamer@checkframetitle% ...revert to it's old definition
  }% Regular \section stuff follows
  \alt#1{\@ifnextchar[\beamer@subsection\beamer@@subsection}
    {\beamer@secgobble}}
\makeatother

% Defining proofs environment for proofs requiring more than one frame
\makeatletter
\newenvironment<>{proofs}[1][\proofname]{%
    \par
    \def\insertproofname{#1\@addpunct{.}}%
    \usebeamertemplate{proof begin}#2}
  {\usebeamertemplate{proof end}}
\makeatother

% Uncomment for some standard notations in math (Real, Complex and Rational numbers, Norm, Jacobian, etc) 
% \newcommand{\reals}{\mathbb{R}}
% \newcommand{\mtrx}{\mathbb{M}}
% \newcommand{\jacobian}{\mathcal{J}}
% \newcommand{\tallstrut}{\vphantom{\frac{5_A}{4,10^3}}}
% \newcommand{\abs}[1]{\left\lvert #1 \right\rvert}
% \newcommand{\norm}[1]{\left\lVert #1 \right\rVert}

%%%%%%%%%%%%%%%%%%%%%%%%%%%%%%%%%%%%%%%%%%%%%%%%%%%%%%%%%%%%%%%%%%%%%%%%%%%%%%
%                 TITLE PAGE, FILL IN FOR PLACEHOLDER TEXT                   %
%%%%%%%%%%%%%%%%%%%%%%%%%%%%%%%%%%%%%%%%%%%%%%%%%%%%%%%%%%%%%%%%%%%%%%%%%%%%%%

% Note: Longer input for Title Page, Abridged Input for Header and Footer

\title[\color{white}El papel del tamaño de los ciclones tropicales en la precipitación en México]{El papel del tamaño de los ciclones tropicales en la precipitación en México
}
\author[Adolfo Pérez]{Adolfo Pérez Estrada}
\institute[PCT UNAM]{Programa de Posgrado en Ciencias de la Tierra \\ Universidad Nacional Autónoma de México}
\date[\insertframenumber/\inserttotalframenumber]{Examen que opta por el grado de: \\ Maestro en Ciencias de la Tierra}
% (Major/Minor Project Presentation, Thesis Defense, etc.)
\titlegraphic{\includegraphics[height=40pt]{Images/Logos/posgrado2.jpg}}

% Document starts here
\begin{document}

\setlength{\belowcaptionskip}{10pt plus 2pt minus 4pt}

% Title page
\begin{frame}[plain]
	\maketitle
\end{frame}

% Chapters
\section{Introducción}
\subsection{Los ciclones tropicales en México}
\begin{frame}
    \begin{figure}[H]
        \centering
        \includegraphics[scale = 0.275]{Images/Figures/Fig_2_1.jpeg}
        \caption{La región de estudio se encuentra marcada por un cuadro negro. Las posiciones de cada 6 horas de los CTs en el Océano Atlántico del norte (NA) y en el Océano Pacífico del este (EP) se encuentran en polígonos rojos y grises, respectivamente}
        \label{fig:fig_1}
    \end{figure}
\end{frame}

\begin{frame}{Escala Saffir-Simpson}
    % Please add the following required packages to your document preamble:

\begin{table}
\centering
\caption{Clasificación de los vientos de los CTs de acuerdo a la escala Saffir-Simpson. Se anexan dos categorías adicionales cuando el CT aún no alcanza la categoría de huracán (Kelman, 2013)}
\label{tab:1.1}
\resizebox{\textwidth}{!}{%
\begin{tabular}{@{}cccc@{}}
\toprule
\multicolumn{2}{c}{\textbf{Categoría}}               & \begin{tabular}[c]{@{}c@{}}\textbf{Vientos Sostenidos} \\ ($km  h^{-1}$)\end{tabular} & \begin{tabular}[c]{@{}c@{}} \textbf{Tipos de daño debido a}\\  \textbf{los vientos del CT}\end{tabular} \\ \midrule
\multicolumn{2}{l}{Depresión Tropical (DT)} & \textless 63                                                           & Daños menores                                                                             \\
\multicolumn{2}{l}{Tormenta Tropical  (TT)}  & 64-118                                                                 & Daños moderados                                                                           \\
\multirow{5}{*}{Huracán}         & 1        & 119-153                                                                & Vientos muy peligrosos                                                                    \\
                                 & 2        & 154-177                                                                & Vientos extremadamente peligrosos                                                         \\
                                 & 3        & 178-208                                                                & Daños devastadores                                                                        \\
                                 & 4        & 209-251                                                                & Daños catastróficos                                                                       \\
                                 & 5        & \textgreater 252                                                       & Daños extraordinarios                                                                     \\ \bottomrule
\end{tabular}%
}
\end{table}
\end{frame}

\begin{frame}
    \begin{figure}
        \centering
        \includegraphics[scale=0.35]{Images/Figures/Fig_1_1.jpeg}
        \caption{Porcentaje de los CTs que hicieron \textit{landfalling} en las costas mexicanas. Los CTs del {\red NA} están en barras rojas  y los {\gray EP} en barras grises.}
        \label{fig:fig_2}
    \end{figure}
\end{frame}

\begin{frame}{El SIAT-CT en México}
    Los desastres históricos asociados al paso de CTs en México es la principal motivación para elaborar planes de acción y mejorar la gestión integral de riesgos. En el año 2000, se propone la creación de un Sistema de Alerta Temprana de Ciclones Tropicales (SIAT-CT) como una herramienta de coordinación entre la población y Protección Civil.

    \begin{block}{Peligro por CT definido por el SIAT-CT}
    \begin{equation}
    \label{eq:1.1}
       e = \displaystyle \frac{(I+C)}{2},
    \end{equation}
    
    $e = $ Peligro \\
    $I = $ Intensidad definida por la Escala Saffir-Simpson \\
    $C = $ Escala de circulación definido por el tamaño del radio de 34 nudos
    \\~\
    \end{block}
\end{frame}

\begin{frame}
\begin{columns}
    \begin{column}{0.35\textwidth}
        \begin{block}{SIAT-CT caso TT Cristobal}
        Representación del nivel de alerta del SIAT-CT para la TT Cristóbal el 4 de junio del 2020 en sus posiciones reportadas durante las 00:00(a), 06:00(b), 12:00(c), 18:00(d) y las 00:00(e) del 5 de junio.
        \end{block}
    \end{column}
    \begin{column}{0.65\textwidth}
    \begin{figure}
        \centering
        \includegraphics[scale = 0.18]{Images/Figures/Fig_1_3.jpeg}
        \label{fig:fig_3}
    \end{figure}
    \end{column}
\end{columns}
\end{frame}

\begin{frame}
    \begin{figure}
        \includegraphics[scale = 0.35]{Images/Figures/Fig_1_2.jpeg}
        \caption{Valores de precipitación reportadas por CHIRPS el día 4 de junio del 2020 en el sureste de México. Las posiciones de la TT Cristóbal se muestran en verde (Tormenta Tropical) y azul (Depresión Tropical)}
        \label{fig:fig_4}  
    \end{figure}
\end{frame}

\subsection{Justificación y Objetivos}
\begin{frame}
    \hfill
    \begin{columns}
    \begin{column}{0.48\textwidth}
    \begin{block}{Justificación}
    Los estudios de CTs en México son limitados, aunque el país es impactado por la actividad ciclónica tropical cada año. Por ello, es necesario mejorar su definición de peligrosidad en el SIAT, con la finalidad de mejorar la gestión integral de riesgo. Es necesario contar con una metodología diseñada para México que defina el tamaño de los CTs.
\\~\ % Used for spacing out the block 
    \end{block}
    \end{column}
    \begin{column}{0.48\textwidth}
    \begin{block}{Objetivo General}
    Definir el tamaño de los ciclones tropicales que afectaron a México usando imágenes satelitales infrarrojas, así como analizar las relaciones estadísticas del tamaño con las variables medioambientales durante el periodo 2000-2020.
\\~\ % Used for spacing out the block 
    \end{block}
    \end{column}
    \end{columns}
\end{frame}

% \\~\ is used to force empty lines to generate to fix general typesetting within blocks.
% \\ = new line and ~\ = empty character
% \lipsum is used to generate dummy text.

\section{Datos y Métodos}

\begin{frame}{El tamaño del CT}
\textbf{¿Qué es el tamaño de los CTs?} \\ El término tamaño del CT suele ser muy ambiguo porque se puede abordar diferentes parametrizaciones considerando principalmente:
\begin{itemize}
    \item La velocidad del viento de los CTs
    \begin{itemize}
        \item Intensidad de los vientos en su circulación interna
        \item Intensidad de los vientos en su circulación externa
    \end{itemize}
    \item La vorticidad
    \item Isobaras (curva de igual o constante presión)
    \item Precipitación reportada a ciertos umbrales
\end{itemize}

\end{frame}

\subsection{Bases de datos utilizadas}
\begin{frame}
    \begin{block}{Para calcular el tamaño}
        \begin{itemize}
            \item Posiciones del CT cada 6 h HURDAT
            \item Productos IR del GPM
            \item Tamaño del campo de vientos.
        \end{itemize}
    \end{block}

    \begin{exampleblock}{Para validar relación entre el tamaño y precipitación}
        \begin{itemize}
            \item Producto satelital GPM\_IMERG
            \item Producto de precipitación CHIRPS
        \end{itemize}
    \end{exampleblock}

    \begin{alertblock}{Para la validación de las variables ambientales con la precipitación}
    Se utilizaron las siguientes variables: 
    Velocidad vertical del viento a 500 hPa, humedad específica a 600 hPa, vorticidad a 200 hPa, divergencia a 200 hPa, cizalladura del viento, contenido total de agua precipitable
    \end{alertblock}
\end{frame}

\subsection{ROCLOUD}
\begin{frame}
    \begin{block}{Sobre el tamaño de la circulación del CT}
        Pérez-Alarcón et al. (2021) desarrollaron una base de datos sobre el tamaño de los CTs utilizando los perfiles uniformes del viento descritos por Willoughby et al. (2006). \bigskip

        Estos cálculos utilizan la posición del CT obtenida del HURDAT2, la intensidad máxima de los vientos y el radio de los vientos máximos, calculado a través de modelos específicos para cada cuenca o en función de la posición latitudinal del CT.

    \begin{equation}
        \label{eq:2.1}
        r_{m} =  46.6 \ \exp{(-0.015 \ V_{max} + 0.0169 \phi)}
    \end{equation}
        \\~\
    \end{block}

\end{frame}

\begin{frame}
    Se diseñó un algoritmo que mide las distancias radiales desde el centro del CT hasta el punto más alejado donde las temperaturas de brillo de las nubes pertenecen a la circulación del CT y están por debajo de -40°C
    \\~\ 
    \\~\ 
\begin{enumerate}
    \item<1-> Segmentación a -40°C
     \\~\
    \item<2-> Selección de regiones de interés
     \\~\
    \item<3-> Selección de polígonos dentro del campo de vientos
     \\~\
    \item<4-> Determinación de los radios por cuadrante
     \\~\
    \item<4-> Cálculo del radio promedio
\end{enumerate}
\end{frame}

\begin{frame}
    \begin{figure}
        \centering
        \includegraphics[scale = 0.32]{Images/Figures/Fig_2_3.jpeg}
        \caption{Extensión del campo de nubes a través de contornos generados con imágenes IR (contornos grises) del huracán Alex 2010, cuyas posiciones están representadas por puntos azules}
        \label{fig:fig_5}
    \end{figure}
\end{frame}

\subsection{Métodos para relacionar la lluvia con el tamaño del ciclón}
\begin{frame}
\begin{enumerate}
\setcounter{enumi}{0}
\item Algoritmo del Radio de las Bandas de Precipitación
(RBP)
      \begin{figure}
            \centering
            \includegraphics[scale = 0.39]{Images/Figures/Fig_2_5.jpeg}
            \caption{Valores de precipitación del GPM\_IMERG para la TT Cristóbal en su posición del 4 de junio del 2020 a las 06:00 UTC.}
            \label{fig:fig_6}
        \end{figure}
\end{enumerate}
\end{frame}

\begin{frame}
\begin{enumerate}
\setcounter{enumi}{1}
\item Técnica de anillos usando los datos de GPM\_IMERG
      \begin{figure}
            \centering
            \includegraphics[scale = 0.4]{Images/Figures/Fig_2_6.jpeg}
            \caption{Similar a la Fig. \ref{fig:fig_6}, pero mostrando los anillos usados cada 50 km.}
            \label{fig:fig_7}
        \end{figure}
\end{enumerate}
\end{frame}

\begin{frame}
\begin{enumerate}
\setcounter{enumi}{2}
\item Técnica de anillos usando los datos continentales de CHIRPS
      \begin{figure}
            \centering
            \includegraphics[scale = 0.24]{Images/Figures/Fig_2_7.jpeg}
            \caption{Posiciones de los CT que se encuentran sobre territorio mexicano (rojo) y cercanos a la costa (distancia a 250 km, en azul).}
            \label{fig:fig_8}
        \end{figure}
\end{enumerate}
\end{frame}

\subsection{Sobre las variables medioambientales y la forma}

\begin{frame}
    \begin{alertblock}{Relaciones estadísticas}
        \begin{enumerate}
            \item Correlación Spearman
            \item Modelos de Estimación de Ecuaciones Generalizadas
        \end{enumerate}
        ~\
    \end{alertblock}
~\
    
    \begin{exampleblock}{La forma del CT}
        \begin{enumerate}
            \item Dispersión
            \item Asimetría
            \item Solidez
        \end{enumerate} 
        ~\
    \end{exampleblock}
\end{frame}

\section{Resultados}

\subsection{Sobre la climatología del tamaño de los CTs}
\begin{frame}
\begin{itemize}
    \item Se analizaron {\red 191} y {\gray 337} CTs de las cuencas {\red NA} y {\gray EP} respectivamente durante el período 2000-2020.
    \\~\
    \item Sólo se consideran las posiciones de CT de 6 horas que se localizan en la región de estudio. Se obtuvieron {\red 4526} y {\gray 6923} posiciones de CTs para las cuencas.
\end{itemize}
\end{frame}

\begin{frame}
    \begin{figure}
        \centering
        \includegraphics[scale = 0.35]{Images/Figures/Fig_3_1.jpeg}
        \caption{Cajas y bigotes de las distribuciones de los radios por cuadrante y el $R_p$ (km) de los radios en la región de estudio de la cuenca {\red NA} y {\gray EP}.}
        \label{fig:fig_9}
    \end{figure}
\end{frame}

\begin{frame}
    \begin{columns}
        \begin{column}{0.4\textwidth}
            \begin{enumerate}
                \setcounter{enumi}{0}
                \item Sobre la variación espacial de los tamaños
            \begin{block}{Figura 9:}
                Distribución espacial del tamaño de los CTs por cuadrante (km): (a) RNE, (b) RNO, (c) RSO, (d) RSE y (e) $R_p$. Los límites en la barra de colores representan los rangos intercuantílicos (p25 y p75) durante el período 2000-2020.
            \end{block}
            \end{enumerate}
        \end{column}
        \begin{column}{0.6\textwidth}
        \begin{figure}
            \centering
            \includegraphics[scale = 0.17]{Images/Figures/Fig_3_6.jpeg}
            \caption{}
            \label{fig:fig_tamaño}
        \end{figure}
        \end{column}
    \end{columns}
\end{frame}

\begin{frame}
    \begin{enumerate}
    \setcounter{enumi}{0}
    \item Sobre la variación espacial de los tamaños
    \begin{figure}
        \centering
        \includegraphics[scale = 0.26]{Images/Figures/Fig_3_8.jpeg}
        \caption{Compuestos de tamaño de Rp (km) de: (a) los CTs sobre {\red NA} y {\gray EP} y las TTs (en amarillo), HUR1-2 (en verde) y HUR3-5 (en azul) sobre (b) {\red NA} y (c) {\gray EP}. El tiempo 0 h representa el momento en que alcanzan la máxima intensidad. Las áreas sombreadas proporcionan el error estándar asociado al valor promedio de $R_p$ cada 6 h.}
        \label{fig:fig_11}
    \end{figure}
    \end{enumerate}
\end{frame}

\begin{frame}
    \begin{enumerate}
    \setcounter{enumi}{1}
        \item Sobre la variación mensual de los tamaños
    \end{enumerate}
    \begin{figure}
        \centering
        \includegraphics[scale = 0.3]{Images/Figures/Fig_3_10.jpeg}
        \caption{Tamaños mensuales de CT para los cuartiles inferior (pequeño) y superior (grande) en: (a) cuencas {\red NA} y (b) {\gray EP}.}
        \label{fig:fig_12}
    \end{figure}
\end{frame}

\subsection{Sobre la relación del tamaño y la precipitación}
\begin{frame}
\begin{enumerate}
\setcounter{enumi}{0}
    \item Climatología de los tamaños usando productos IMERG
\end{enumerate}
    \begin{figure}
        \centering
        \includegraphics[scale = 0.3]{Images/Figures/Fig_3_13.jpeg}
        \caption{Cajas y bigotes de las distribuciones de los radios por cuadrante y el $R_p$ (km) de los radios en la región de estudio de la cuenca {\red NA} y {\gray EP} definidos por el algoritmo RBP.}
        \label{fig:fig_13}
    \end{figure}
\end{frame}

\begin{frame}
    \begin{exampleblock}{Para el NA}
        % Please add the following required packages to your document preamble:
% \usepackage{booktabs}
% \usepackage{graphicx}
\begin{table}[H]
\centering
\caption{Correlación de rango de Spearman (r) entre los tamaños calculados con las técnicas de ROCLOUD y RBP para cada cuadrante y promedio; Se determinan para la cuenca NA. Los valores con un asterisco son significativos a un nivel del 95\% de confianza.}
\label{tab:3.5}
\resizebox{\textwidth}{!}{%
\begin{tabular}{@{}llllll@{}}
\toprule
             & RNE\_RPB       & RNO\_RPB       & RSO\_RPB       & RSE\_RPB       & Rp\_RPB        \\ \midrule
RNE\_ROCLOUD & \textit{0.81*} &                &                &                &                \\
RNO\_ROCLOUD & \textit{0.14}  & \textit{0.67*} &                &                &                \\
RSO\_ROCLOUD & \textit{0.05}  & \textit{0.17}  & \textit{0.63*} &                &                \\
RSE\_ROCLOUD & \textit{0.32*} & \textit{0.09}  & \textit{0.18}  & \textit{0.66*} &                \\
Rp\_ROCLOUD  & \textit{0.59*} & \textit{0.32*} & \textit{0.37*} & \textit{0.5*}  & \textit{0.73*} \\ \bottomrule
\end{tabular}%
}
\end{table}
    \end{exampleblock}
\end{frame}

\begin{frame}
    \begin{alertblock}{Para el EP}
        % Please add the following required packages to your document preamble:
% \usepackage{booktabs}
% \usepackage{graphicx}
\begin{table}[H]
\centering
\caption{Igual a la Tabla \ref{tab:3.5}, pero para el EP}
\label{tab:3.6}
\resizebox{\textwidth}{!}{%
\begin{tabular}{@{}llllll@{}}
\toprule
             & RNE\_RPB       & RNO\_RPB       & RSO\_RPB       & RSE\_RPB       & Rp\_RPB        \\ \midrule
RNE\_ROCLOUD & \textit{0.77*} &                &                &                &                \\
RNO\_ROCLOUD & \textit{0.32*} & \textit{0.71*} &                &                &                \\
RSO\_ROCLOUD & \textit{0.12}  & \textit{0.21}  & \textit{0.69*} &                &                \\
RSE\_ROCLOUD & \textit{0.26}  & \textit{0.08}  & \textit{0.28}  & \textit{0.67*} &                \\
Rp\_ROCLOUD  & \textit{0.56*} & \textit{0.38*} & \textit{0.48*} & \textit{0.57*} & \textit{0.74*} \\ \bottomrule
\end{tabular}%
}
\end{table}
    \end{alertblock}
\end{frame}

\begin{frame}
\begin{enumerate}
\setcounter{enumi}{1}
    \item Dependencia de la PCT con el tamaño del CT: GPM\_MERGIR
\end{enumerate}
    \begin{columns}
        \begin{column}{0.6\textwidth}
        \begin{figure}
            \centering
            \includegraphics[scale = 0.215]{Images/Figures/Fig_3_16.jpeg}
            \caption{}
            \label{fig:fig_rbp}
        \end{figure}
        \end{column}
        
        \begin{column}{0.4\textwidth}
            \begin{block}{Figura 13:}
                Tasa de precipitación ($mm h^{-1}$) de IMERG del CT en función del radio (km), medido por la técnica de anillos, para (a) todas las posiciones, (b) posiciones sobre continente y (c) posiciones que se encuentren al menos a 250 km de la costa.
            \end{block}
        \end{column}
    \end{columns}
\end{frame}

\begin{frame}
\begin{enumerate}
\setcounter{enumi}{2}
    \item Dependencia de la PCT con el tamaño del CT: CHIRPS
\end{enumerate}
    \begin{figure}
        \centering
        \includegraphics[scale = 0.35]{Images/Figures/Fig_3_21.jpeg}
        \caption{Tasa de precipitación (mm $h^{-1}$) del CTs en función de su tamaño (km), definido por los rangos intercuartílicos del tamaño definido por ROCLOUD, en las cuencas: (a) {\red NA} y (b) {\gray EP}.}
        \label{fig:figchirps}
    \end{figure}
\end{frame}

\subsection{Sobre las variables medioambientales y la precipitación}
\begin{frame}
% Please add the following required packages to your document preamble:
% \usepackage{booktabs}
% \usepackage{graphicx}
% \usepackage[table,xcdraw]{xcolor}
% If you use beamer only pass "xcolor=table" option, i.e. \documentclass[xcolor=table]{beamer}
\begin{table}[H]
\centering
\caption{Coeficientes y estimación de modelos de la lluvia acumulada diaria en el área del CT para la cuenca NA y EP. Unidad de PCP: $10^{3}$ mm.}
\label{tab:3.8}
\resizebox{\textwidth}{!}{%
\begin{tabular}{@{}ccc@{}}
\toprule

{\color[HTML]{000000} \textit{\textbf{Variables}}} & {\color[HTML]{000000} \textit{\textbf{Coeficientes del modelo NA}}} & {\color[HTML]{000000} \textit{\textbf{Coeficientes del modelo EP}}} \\ \midrule
 
{\color[HTML]{000000} Intersección con el eje}     & {\color[HTML]{000000} 13.66}                                        & {\color[HTML]{000000} -17.45}                                       \\

{\color[HTML]{000000} $V_{max}$}                        & {\color[HTML]{000000} 0.01}                                        & {\color[HTML]{000000} 0.02}                                        \\
 
{\color[HTML]{000000} $R_p$}                          & {\color[HTML]{000000} 0.23}                                        & {\color[HTML]{000000} 0.22}                                        \\

{\color[HTML]{000000} \textit{PNM}}                         & {\color[HTML]{000000} -0.02}                                        & {\color[HTML]{000000} -0.02}                                       \\
 
{\color[HTML]{000000} \textit{HE}}                          & {\color[HTML]{000000} 21.14}                                        & {\color[HTML]{000000} 20.89}                                        \\

{\color[HTML]{000000} \textit{CIZ}}                         & {\color[HTML]{000000} 0.42}                                        & {\color[HTML]{000000} 0.20}                                        \\

{\color[HTML]{000000} \textit{DIV}}                         & {\color[HTML]{000000} 180.4}                                        & {\color[HTML]{000000} 186.4}                                       \\

{\color[HTML]{000000} Número de casos}             & {\color[HTML]{000000} 4526}                                         & {\color[HTML]{000000} 6993}                                         \\
{\color[HTML]{000000} RMSE}                        & {\color[HTML]{000000} 0.529}                                        & {\color[HTML]{000000} 0.498}                                        \\
{\color[HTML]{000000} QIC}                         & {\color[HTML]{000000} 453}                                          & {\color[HTML]{000000} 699}                                          \\ \bottomrule
\end{tabular}%
}
\end{table}
\end{frame}

\subsection{Sobre la forma del CT}
\begin{frame}
\begin{figure}
    \centering
    \includegraphics[scale = 0.3]{Images/Figures/Fig_3_26_1.jpeg}
    \caption{Distribución espacial en una malla de 2 $\times$ 2°de la métrica de $D$ de los CTs analizados. Los rangos de color en la escala fueron determinados por los rangos intercuartílicos}
    \label{fig:fig_10}
\end{figure}
\end{frame}

\subsection{Sobre la forma del CT}
\begin{frame}
\begin{figure}
    \centering
    \includegraphics[scale = 0.25]{Images/Figures/Fig_3_27.jpeg}
    \caption{Como en la Fig. \ref{fig:fig_10}, pero para $A$}
    \label{fig:fig_A}
\end{figure}

\begin{figure}
    \centering
    \includegraphics[scale = 0.25]{Images/Figures/Fig_3_28.jpeg}
    \caption{Como en la Fig. \ref{fig:fig_10}, pero para $S$}
    \label{fig:fig_S}
\end{figure}

\end{frame}
\section{Conclusiones y Trabajo Futuro}
\begin{frame}{Conclusiones}
    \begin{enumerate}
        \item<1-> El tamaño del CT calculado con el algoritmo ROCLOUD muestra ser operacionalmente funcional. Otros estudios similares no tienen en cuenta la precipitación del CT. 
        \\~\
        \item<2-> Además, la intensidad del CT no está relacionada con su tamaño. Por ejemplo, los TD pueden tener un tamaño mayor que los huracanes.
        \\~\
        \item<3-> La precipitación del CT calculada a partir de dos enfoques (radio de las bandas de lluvia del CT y perfiles pcp) muestra que los CT producen importantes precipitaciones en regiones remotas situadas a ~750 km del centro del CT. Es  mayor sobre las regiones continentales ya que las bandas de lluvia están más dispersas que sobre las regiones oceánicas.
    \end{enumerate}
\end{frame}

\begin{frame}{Conclusiones}
    \begin{enumerate}
    \setcounter{enumi}{3}
        \item<1->Las variables a gran escala desempeñan un papel importante en la determinación de las precipitaciones del CT.
        \\~\
        \item<2->La humedad a niveles medios, la cizalladura del viento y la divergencia en la atmósfera superior muestran una fuerte relación con la precipitación del CT. 
        \\~\
        \item<3->Finalmente, las métricas de forma Dispersión ($D$), Asimetría ($A$) y Solidez ($S$) muestran que, en promedio, los CTs de ambas cuencas tienden a ser más dispersos, asimétricos y menos sólidos.
    \end{enumerate}
\end{frame}

\begin{frame}
\begin{columns}
    \begin{column}{0.5\textwidth}
        \begin{block}{\textbf{Sobre el SIAT-CT}}
        La motivación principal de este trabajo radica en las limitaciones que tiene el SIAT-CT en términos de la definición del tamaño. Los resultados muestran que la influencia de la PCT puede alcanzar los 750 km desde el centro del CT. Por ello, se vuelve relevante usar una definición diferente al R34, que incorpore un tamaño del CT que considere la extensión de las bandas nubosas del CT y su precipitación.
        \end{block}
    \end{column}
    \begin{column}{0.5\textwidth}
    \begin{figure}
        \centering
        \includegraphics[scale = 0.135]{Images/Figures/Fig_4_1.jpeg}
        \label{fig:my_label}
    \end{figure}
    \end{column}
\end{columns}
\end{frame}

\begin{frame}{Trabajo Futuro}
    \begin{itemize}
        \item Estudiar posibles mejoras en las alertas tempranas modificando los tamaños del CT.
        \\~\
        \item Usar técnicas de machine learning para predecir el tamaño de un CT, con fines operativos.
        \\~\
        \item Estudiar el cambio climático y su papel en el tamaño de los CTs en un futuro.
    \end{itemize}
\end{frame}


% This section is a placeholder for you to go over crucial points to takeaway from your presentation.

\begin{frame}
\begin{center}
\Huge ¡Muchas Gracias!
\end{center}
\end{frame}

% Bibliography/References
%\begin{frame}[allowframebreaks]
%	\frametitle{References}
%	\nocite{*}
%	\printbibliography
%\end{frame}

\end{document}